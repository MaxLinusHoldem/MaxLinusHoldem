\documentclass[10pt,twoside,a4paper]{article}

\usepackage[T1]{fontenc}
\usepackage[utf8]{inputenc}
\usepackage[swedish]{babel}

\usepackage{fancyhdr}

\pagestyle{fancy}
\fancyhead{}
\fancyfoot{}

\fancyhead[L]{Linus Wåreus\\Max Wällstedt\\DD1339 Introduktion till datalogi\\\textbf{Projekt}}
\fancyhead[R]{Grupp 3\\2013/05/15\\}

\setlength{\headheight}{47pt}

\begin{document}
\section*{Slutrapport}

\subsection*{Programbeskrivning}

\subsubsection*{Del A}

I vårt projekt har vi tänkt göra ett pokerspel. Planen är helt enkelt att
utveckla ett program där man kan spela poker, närmare bestämt Texas Hold
’em no-limit. Programmet ska i ett första skede bestå av en klient där man
kan köra en spelare mot datorn, som kan vara flera spelare samtidigt.
Programmet ska då visa upp en grafisk spelplan där man kan se motståndarna,
jackpotten och ens egna kort och marker. När man sedan spelar spelet ska
det sedan vara som att spela Texas Hold ’em i verkligheten. Man skall kunna
se sina kort, kunna göra insatser, checka, syna och folda. Användaren
kommer då med andra ord ha diverse val huruvida denne skall satsa ens
marker. I Texas Hold’em kan man inte påverka ens kort på något sätt så de
enda knappvalen som kommer behöva finnas är relaterade till hur man satsar
ens marker. Följande knappar bör då finnas, checka, göra en insats, syna en
insats, göra en reraise och folda. När användaren lägger en insats eller
gör en reraise så ska denne även kunna bestämma hur mycket denna insats
skall vara.

I ett nästa steg av programmet vill vi även utveckla en möjlighet att spela
flera spelare genom att göra ett LAN-spel. Då ska helt enkelt en
server-version av programmet köras på en dator och sedan skall spelare
kunna ansluta till servern för att kunna spela flera spelare. Man ska även
i denna version kunna ha med dator-spelare i spelet. Denna version kommer
alltså att se ut precis som den tidigare versionen, med undantag att flera
spelare skall kunna delta i ett spel över flera datorer. Vår plan är helt
enkelt att utveckla spelet i två steg, en offline-version och en
online-version. Vi kommer att börja med att utveckla offline-version, dock
kommer den vara förberedd för att göras om till en online-version. Vi är
inte helt säkra på att vi kommer att lyckas att utveckla en online-version,
dock så kommer offline-version att räcka för att vara ett färdigt program
att presentera. Online-version en kan alltså ses som ett extra arbete som
vi kommer försöka att utveckla i mån av tid och om det inte är allt för
svårt.

\subsection*{Användarbeskrivning}

\subsubsection*{Del A}

Nästa fråga är då vem som kommer att använda vårt program. Vi har tänkt oss
att man som spelare måste ha förkunskaper om hur man spelar Texas Hold ’em
för att kunna spela vårt spel. Man kommer dock kunna spela spelet utan
dessa förkunskaper, men man kommer då behöva lära sig spelet efter hand man
spelar det. Vår utgångspunkt är dock att alla som spelar vårt spel vet sen
innan vet hur Texas Hold ’em går till. Däremot tänker vi oss att spelaren
inte behöver vara en särskilt van datoranvändare för att kunna spela
spelet. Menyval och andra knappar skall vara så pass intuitiva så att även
en relativt ovan datoranvändare skall kunna spela spelet, så länge denne
vet hur Texas Hold ’em går till. Användaren bör ändå ha en viss datorvana
för att kunna spela, den vanan behöver dock inte vara särskilt stor. Vad
det gäller ålder på spelaren så ska det i vårt fall inte spela någon roll,
så länge som sagt spelaren vet hur man spelar Texas Hold ’em. Man kan då
dock göra ett antagande om att spelaren åtminstone är över 10 år, då yngre
barn förmodligen inte vet hur man spelar poker. Så programmet ska med andra
ord kunna spelas av tio åringar och någon övre åldersgräns finns inte. Vår
huvudsakliga målgrupp kommer dock vara vuxna personer, fast spelet skall
ändå kunna spelas av barn äldre en tio år.

\subsection*{Användarscenarier}

\subsubsection*{Del A}

Scenario 1:

\vspace{1em}
\noindent
Karl, 20 år är en van datoranvändare och är också väl insatt i hur man
spelar Texas Hold ’em. Karl börjar med att starta spelet, han ser direkt
att en knapp där det står start nytt spel. Han matar in sitt namn och även
hur många motståndare han vill spela emot, han väljer maximalt antal.
Spelet startar och dealern blandar kortleken och delar ut korten. Karl är
lilla mörker och satsar då automatiskt 5 dollar. Spelet börjar och två
spelare väljer att folda och resten av de datorstyrda spelarna väljer att
syna den stora mörkern på 10 dollar. När det blir Karls tur ser han att han
fått ruter knekt och spader kung. Han väljer att också syna den stora
mörkern och spelet försätter med att dealern delar ut de tre första
gemensamma korten, floppen. Den visar spader tio, hjärter ess och ruter
två. Det är Karls tur igen att börja, eftersom han bara saknar en dam för
att få en stege så väljer han att göra en insats. På skärmen framför sig
ser han knapparna Check, Raise och Fold. Han klickar på Raise. Då dyker ett
nytt alternativ upp med en ruta där det står 5 dollar och sedan en nedåtpil
till vänster om rutan och en uppåtpil till höger om rutan. Han kan inte
klicka på nedåtpilen eftersom att 5 dollar är minsta insatsen. Det finns
även en knapp där det står ”All-in” som betyder att man satsar alla ens
marker. Karl väljer dock att trycka på uppåtpilen tills det står 20 dollar
och sedan klickar på Raise-knappen igen. Spelet fortsätter och alla utom
två spelare foldar, de andra två synar Karls insats. Dealern delar sedan ut
det fjärde kortet, spader 4. Karl väljer denna gång att bara checka. Dock
så höjer en av de andra spelarna insatsen med 20 dollar. När det blir Karls
tur igen så har han tre alternativ på skärmen, Call, Reraise och Fold. Han
väljer call vilket gör att han satsar den summan som krävs för att spelet
skall gå vidare, alltså 20 dollar. Sedan delar dealern ut det sista och
avgörande kortet, the river som visar sig vara spader dam. Detta betyder
att Karl har en steg. Han börjar den sista omgången och återigen höjer han
med 20 dollar. Denna gång foldar en av de andra spelarna och den andre
synar hans insats. Karl börjar med att visa hans kort och spelet
presenterar att han har en stege. Sedan visas datorns kort, denne hade ett
tvåpar i tvåor och damer. Detta gör att Karl vinner och på skärmen står det
”Karl wins” och även hur mycket han vann. Karl känner sig nöjd efter detta
och väljer att avsluta spelet genom att klicka ”Exit” från menyn i toppen
av spelfönstret varpå spelet avslutas.

\vspace{1em}
\noindent
Scenario 2:

\vspace{1em}
\noindent
Bertil, 55 år är inte en jättevan datoranvändare, men han vet hur man
spelar Texas Hold ’em. Han börjar med att starta spelet och möts av en
välkomstskärm som presenterar spelet. Han ser tre val, Single player,
Mulitplayer och Exit. Han väljer multiplayer. Han möts sedan av valen Host
game och Join game. Han väljer Join game. Han blir då uppmanad att mata in
sitt namn och sedan finns det en ruta som visar alla spel som finns i det
lokala nätverket. Det finns även en ruta till höger där man kan mata in
IP-numret till en spelserver. Han ser att det finns ett lokalt spel i den
ena rutan och klickar på den och sedan på knappen ”Join”. Spelet laddar och
på skärmen står det ”Awaiting opponents”. Efter att ha väntat cirka 30
sekunder startar spelet. I spelet finns fyra motståndare till Bertil.
Spelet startar och dealern delar ut korten. Bertil får en spader två och en
hjärter sju. Bertil är den tredje spelaren att satsa och väljer att möta
den stora mörkern på 10 dollar. Alla spelare är kvar i spelet och dealer
delar ut floppen. Floppen gav inte Bertil några bra möjligheter och han
sitter inte på något. Den första spelaren väljer att höja med 50 dollar och
när det blir Bertils tur så är valet självklart. På hans skärm ser han sina
kort, hur mycket marker hand har och även alternativen, Call, Reraise och
Fold. Bertil väljer att folda och spelet fortsätter utan honom. Efter det
att omgången är klar flyttas dealerbrickan och en ny spelomgång tar vid.
Bertil tycket inte det var roligt att spela mot andra riktiga spelare så
han vill hellre spel mot datorn. I menyn i toppen av fönstret väljer därför
Bertil valet ”Disconnect from server”. När han klickar där kommer han
tillbaka till huvudmeny och välkomstskärmen. Där väljer han istället valet
”Single player” och startar upp ett nytt spel mot datorn.

\subsection*{Testplan}

\subsubsection*{Del A}

Användartestning kommer att ske då vi har en tillräckligt fungerande
prototyp. När vi har detta tänker vi finna lämpligt antal personer som vi
kan observera när de använder programmet. Detta personer bör kunna spela
Texas Hold 'em för att få vettig data ur observationen.

Vi bör inte ge ledning åt användarna under testningen, utan låta dem fritt
utforska gränssnittet. Det viktigaste att observera är om gränssnittet
intuitivt kan användas så att det inte påverkar spelupplevelsen.

Exakt när vi utför användartestning kan vi inte säga nu. Vi kan dock säga
att eftersom multiplayervarianten är lite osäker om den blir av så kommer
vi förmodligen inte få tillfälle i alla fall att användartesta den. Det är
bättre att snabbt kunna användartesta singleplayervarianten och få relevant
feedback så vi kan bygga ett bra gränssnitt mot spelet. Detta gränssnitt
kommer även att användas i multiplayervarianten, så all testning kommer att
vara användbar även här.

Självklart ska alla publika metoder i klasser som inte involverar grafiken
enhetstestas så tidigt som möjligt. Ideellt skrivs enhetstester samtidigt
som klasserna implementeras.

\subsubsection*{Del B}

Linus skrev testklasser till Deck och Evaluate, och allt grafiskt testades
manuellt med hjälp av utskrifter i terminalen och dialogfönster.

I användartestningen frågade vi oberoende slumpvis valt folk att testa
spelet. Vi presenterade kort spelet och bad dem att ''tänka högt'' under
spelomgången. Den respons vi fick var framför allt på de buggar som vi inte
hade fixat än eller funktioner som inte var fullt implementerade än. I
övrigt verkade det som att de tyckte att spelet var roligt och snyggt.

\subsection*{Programdesign}

Klasstrukturen i den här typen av spel är ganska självklar i många
avseenden. Exempelvis behövs klasser för kort, användare och även statiska
klasser som att evaluera spelarnas händer och bestämma vinnare. Spelet i
sig bör även finnas i en klass som håller koll på turordning och sköter
varje spelares interaktion med själva spelet, som att tilldela spelare kort
i början av en ny runda och hålla koll på vad varje spelare satsar.

Grafiken bör vara i sina egna klasser i den mån det går för att hålla
programstrukturen modulär. Gränssnittsbyggande klassen bör även vara den
klass som innehåller main-metoden.

I en eventuell multiplayervariant hoppas vi att programstrukturen kan vara
nästan identisk med singelplayervarianten. Serverhosting och
nätverkskommunikation får ske i sin klass, och under spelet kommer
värddatorn köra spelklassen som håller koll på spelarnas interaktion med
spelet. Lokalt på klientdatorerna kommer framför allt grafiken ritas. Alla
datorer som är anslutna till spelet kommer därför regelbundet behöva få
information om vad som har hänt i spelet.

\subsection*{Tekniska frågor}

Tekniska frågor i första stadiet handlar mest om hur vi ska göra
gränssnittet och få detta att kommunicera med spelet. Det mesta vad gäller
själva spelet är ganska straight forward och bara att börja jobba på. Mer
specifik än så är svår att vara i detta skede.

I en eventuell multiplayervariant finns dock större tekniska frågor, dock
även här ganska allmänna eftersom vi inte har så mycket erfarenhet inom
detta område, så det är även här svårt att vara särskilt specifik. När vi
har kommit igång med kodandet kommer tekniska frågor bli mer konkreta och
aktuella.

\subsection*{Arbetsplan}

I vårt projekt planerar vi att dela upp arbetet jämnt mellan varandra. När
vi sedan kommit överens om vilka olika delar vi kommer programmera av
programmet så kommer vi att arbeta relativt självständig. Vi kommer dock
med jämna mellanrum att stämma av varandra och ha möten för att synka
arbetsgången. Vi kommer också självfallet att ge varandra den kod man har
kodat så fort den är klar, så att den andra personen kan använda den koden
för att gå vidare i projektet. Vi hade tänkt dela upp programmet i två
delar, själva spelmotorn och sedan det grafiska. Uppdelningen av arbetet
kommer sedan att bli så att vi kodar lika mycket var på båda dessa delar,
så att inte bara en person gör till exempel den grafiska delen. Exakt hur
denna uppdelning kommer att ske är inte helt klart, men kommer att ske
under arbetets gång.

Till att börja med har vi tänkt göra en prototyp där bara en spelare kan
spela mot datorn. Denna prototyp skall dock vara ganska komplett på så sätt
att den kommer ha ett färdigt grafiskt gränssnitt. När sedan denna version
är klar så tänkte vi, i mån av tid, att utveckla programmet så att man kan
spela flera spelare över det lokala nätverket. Vi kan dock inte i nuläget
säga att denna version kommer att lyckas då vi båda inte har några
erfarenheter när det gäller att koda Java-program som har
nätverksfunktioner. Vi kommer dock att försöka ta till oss den
informationen som krävs för att vi skall lyckas med detta och
förhoppningsvis som kommer vi lyckas med att göra en
multiplayerimplementation.

Vad gäller tidsplanen för projektet så är det första som skall lämnas in
detta dokument fredag den 2 maj. Vi hade även tänkt att till dess börja
koda lite smått på programmet. Den första muntliga lägesrapporten kommer
Max presentera och den andra Linus. Vad det gäller den sista lägesrapporten
har vi inte kommit överens om vem som skall presentera den. När det sedan
kommer till programmeringsbiten så hade vi tänkt börja starta ordentlig
andra veckan. Tills övningen på fredag den 9 maj så har vi som mål att ha
en första spelbar prototyp. Kanske inte en helt fullständigt sådan, men
gärna så fullständig som möjlig. Denna prototyp bör också ha något form av
grafiskt gränssnitt, så vi måste med andra ord börja med det grafiska den
veckan också. Den sista veckan hade vi tänkt finputsa prototypen och även
starta med att koda multiplayerläget, om detta inte redan gjorts. Sista
veckan skall vi även skriva rapporten som skall lämnas in och även göra
användartesterna som tillhör. Allt detta bör göras tidigt den tredje veckan
så att man har onsdagen och torsdagen till att göra finjusteringar på
spelet och se till så att allt är klart för inlämning. Sedan skall allt
lämnas in den sista övningen fredag den 16 maj.

\end{document}
